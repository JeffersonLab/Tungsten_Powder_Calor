
\section{Conclusions}
In this paper we describe a new type of compact calorimeter allowing to develop large scale detectors of similar type, and with certain possible improvements. One of the features of described calorimeter is having possibility of using smaller diameter fibers at the same sampling ratio that would increase the homogeneity of the detector and provide better energy resolution. As a radiator can be used same Tungsten powder that has been used in presented work. The problem of providing uniform distribution of fibers in the working volume should not be much difficult to solve taking into account results already obtained in the existing prototype. In this case one would expect the density of radiator above density of metal lead while whole assembly build without any air gaps between fibers and radiator.

\section{Acknowledgements}
This material is based upon work supported by the US Department of Energy, Office of Science, Office of Nuclear Physics under contracts DE-AC05-06OR23177, and by the National Aeronautics and Space Administration (NASA) grant No. FAR-NAGS-8653.
\newline

We appreciate and are very thankful to Nick Markov and Isabella Illari for numerous valuable discussions and interesting comments encouraging future developments. Special thanks to Izzy Illari for her work in preparation of the paper for publication.
